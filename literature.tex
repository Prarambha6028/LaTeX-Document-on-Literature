\documentclass{book}
\usepackage{graphicx} % Required for inserting images

\title{Literature}
\author{Prarambha Barman}
\date{\today}

\begin{document}
	
	\maketitle
	\chapter*{Preface}
	\paragraph{}
	This is for the LaTex course that i am currently doing on edX. I am writing this LaTeX document on Literature, the information is taken from Wikipedia.
	Link: https://en.wikipedia.org/wiki/Literature
	
	\chapter*{About Me}
	\paragraph{}
	Hello, i am Prarambha Barman, 20 years old from West Bengal. I am currently pursuing a Bachelor's degree in Computer Science. I am very interested in Mathematics, I was always weak in that subject from childhood but in 11th standards, when I took Science in 11th grade, I started developing an interest in Math. Chemistry, too, intrigued me. I am currently trying to learn new skills like C/C++, PyThon, and other programming languages,  Matlab, Embedded System, and of course LaTeX.
	
	\tableofcontents
	\chapter{Literature}
	\section{Introduction}
	\paragraph{}
	Literature is any collection of written work, but it is also used more narrowly for writings specifically considered to be an art form, especially novels, plays, and poems. It includes both print and digital writing. In recent centuries, the definition has expanded to include oral literature, much of which has been transcribed. Literature is a method of recording, preserving, and transmitting knowledge and entertainment. It can also have a social, psychological, spiritual, or political role.
	
	\paragraph{}
	Literary criticism is one of the oldest academic disciplines, and is concerned with the literary merit or intellectual significance of specific texts. The study of books and other texts as artifacts or traditions is instead encompassed by textual criticism or the history of the book. "Literature", as an art form, is sometimes used synonymously with literary fiction, fiction written with the goal of artistic merit, but can also include works in various non-fiction genres, such as biography, diaries, memoirs, letters, and essays. Within this broader definition, literature includes non-fictional books, articles, or other written information on a particular subject.
	
	Developments in print technology have allowed an ever-growing distribution and proliferation of written works, while the digital era has blurred the lines between online electronic literature and other forms of modern media.
	
	\section{Definition}
	\paragraph{}
	Definitions of literature have varied over time. In Western Europe, prior to the 18th century, literature denoted all books and writing. It can be seen as returning to older, more inclusive notions, so that cultural studies, for instance, include, in addition to canonical works, popular and minority genres. The word is also used in reference to non-written works: to "oral literature" and "the literature of preliterate culture".
	
	Etymologically, the term derives from Latin literatura/litteratura, "learning, writing, grammar," originally "writing formed with letters," from litera/littera, "letter." In spite of this, the term has also been applied to spoken or sung texts. Literature is often referred to synecdochically as "writing," especially creative writing, and poetically as "the craft of writing" (or simply "the craft"). Syd Field described his discipline, screenwriting, as "a craft that occasionally rises to the level of art."
	
	A value judgment definition of literature considers it as consisting solely of high quality writing that forms part of the belles-lettres ("fine writing") tradition. An example of this is in the 1910–1911 Encyclopædia Britannica, which classified literature as "the best expression of the best thought reduced to writing".
	
	\chapter{History}
	\section{Literature History}
	\paragraph{}
	The use of the term "literature" here poses some issues due to its origins in the Latin littera, "letter," essentially writing. Alternatives such as "oral forms" and "oral genres" have been suggested, but the word literature is widely used.
	
	Australian Aboriginal culture has thrived on oral traditions and oral histories passed down through tens of thousands of years. In a study published in February 2020, new evidence showed that both Budj Bim and Tower Hill volcanoes erupted between 34,000 and 40,000 years ago. Significantly, this is a "minimum age constraint for human presence in Victoria", and also could be interpreted as evidence for the oral histories of the Gunditjmara people, an Aboriginal Australian people of south-western Victoria, which tell of volcanic eruptions being some of the oldest oral traditions in existence. An axe found underneath volcanic ash in 1947 had already proven that humans inhabited the region before the eruption of Tower Hill.
	
	Oral literature is an ancient human tradition found in "all corners of the world."[18] Modern archaeology has been unveiling evidence of the human efforts to preserve and transmit arts and knowledge that depended completely or partially on an oral tradition, across various cultures:
	
	The Judeo-Christian Bible reveals its oral traditional roots; medieval European manuscripts are penned by performing scribes; geometric vases from archaic Greece mirror Homer's oral style. (...) Indeed, if these final decades of the millennium have taught us anything, it must be that oral tradition never was the other we accused it of being; it never was the primitive, preliminary technology of communication we thought it to be. Rather, if the whole truth is told, oral tradition stands out as the single most dominant communicative technology of our species as both a historical fact and, in many areas still, a contemporary reality.
	
	The earliest poetry is believed to have been recited or sung, employed as a way of remembering history, genealogy, and law.
	
	In Asia, the transmission of folklore, mythologies as well as scriptures in ancient India, in different Indian religions, was by oral tradition, preserved with precision with the help of elaborate mnemonic techniques.
	
	The early Buddhist texts are also generally believed to be of oral tradition, with the first by comparing inconsistencies in the transmitted versions of literature from various oral societies such as the Greek, Serbia and other cultures, then noting that the Vedic literature is too consistent and vast to have been composed and transmitted orally across generations, without being written down. According to Goody, the Vedic texts likely involved both a written and oral tradition, calling it a "parallel products of a literate society".
	
	All ancient Greek literature was to some degree oral in nature, and the earliest literature was completely so. Homer's epic poetry, states Michael Gagarin, was largely composed, performed and transmitted orally. As folklores and legends were performed in front of distant audiences, the singers would substitute the names in the stories with local characters or rulers to give the stories a local flavor and thus connect with the audience by making the historicity embedded in the oral tradition as unreliable. The lack of surviving texts about the Greek and Roman religious traditions have led scholars to presume that these were ritualistic and transmitted as oral traditions, but some scholars disagree that the complex rituals in the ancient Greek and Roman civilizations were an exclusive product of an oral tradition.
	
	Writing systems are not known to have existed among Native North Americans (north of Mesoamerica) before contact with Europeans. Oral storytelling traditions flourished in a context without the use of writing to record and preserve history, scientific knowledge, and social practices. While some stories were told for amusement and leisure, most functioned as practical lessons from tribal experience applied to immediate moral, social, psychological, and environmental issues. Stories fuse fictional, supernatural, or otherwise exaggerated characters and circumstances with real emotions and morals as a means of teaching. Plots often reflect real life situations and may be aimed at particular people known by the story's audience. In this way, social pressure could be exerted without directly causing embarrassment or social exclusion. For example, rather than yelling, Inuit parents might deter their children from wandering too close to the water's edge by telling a story about a sea monster with a pouch for children within its reach.[30]
	
	The enduring significance of oral traditions is underscored in a systemic literature review on indigenous languages in South Africa, within the framework of contemporary linguistic challenges. Oral literature is crucial for cultural preservation, linguistic diversity, and social justice, as evidenced by the postcolonial struggles and ongoing initiatives to safeguard and promote South African indigenous languages.
	\subsection{Oratory}
	\paragraph{}
	Oratory or the art of public speaking was considered a literary art for a significant period of time. From ancient Greece to the late 19th century, rhetoric played a central role in Western education in training orators, lawyers, counselors, historians, statesmen, and poets.
	\subsection{Writing}
	\paragraph{}
	Around the 4th millennium BC, the complexity of trade and administration in Mesopotamia outgrew human memory, and writing became a more dependable method of recording and presenting transactions in a permanent form. Though in both ancient Egypt and Mesoamerica, writing may have already emerged because of the need to record historical and environmental events. Subsequent innovations included more uniform, predictable legal systems, sacred texts, and the origins of modern practices of scientific inquiry and knowledge-consolidation, all largely reliant on portable and easily reproducible forms of writing.
	\subsection{Early written literature}
	\paragraph{}
	Ancient Egyptian literature, along with Sumerian literature, are considered the world's oldest literatures.[36] The primary genres of the literature of ancient Egypt—didactic texts, hymns and prayers, and tales—were written almost entirely in verse; By the Old Kingdom (26th century BC to 22nd century BC), literary works included funerary texts, epistles and letters, hymns and poems, and commemorative autobiographical texts recounting the careers of prominent administrative officials. It was not until the early Middle Kingdom (21st century BC to 17th century BC) that a narrative Egyptian literature was created.
	
	Many works of early periods, even in narrative form, had a covert moral or didactic purpose, such as the Sanskrit Panchatantra (200 BC – 300 AD), based on older oral tradition.[39][40] Drama and satire also developed as urban cultures, which provided a larger public audience, and later readership for literary production. Lyric poetry (as opposed to epic poetry) was often the speciality of courts and aristocratic circles, particularly in East Asia where songs were collected by the Chinese aristocracy as poems, the most notable being the Shijing or Book of Songs (1046–c.600 BC).
	
	In ancient China, early literature was primarily focused on philosophy, historiography, military science, agriculture, and poetry. China, the origin of modern paper making and woodblock printing, produced the world's first print cultures. Much of Chinese literature originates with the Hundred Schools of Thought period that occurred during the Eastern Zhou dynasty (769‒269 BC). The most important of these include the Classics of Confucianism, of Daoism, of Mohism, of Legalism, as well as works of military science (e.g. Sun Tzu's The Art of War, c.5th century BC) and Chinese history (e.g. Sima Qian's Records of the Grand Historian, c.94 BC). Ancient Chinese literature had a heavy emphasis on historiography, with often very detailed court records. An exemplary piece of narrative history of ancient China was the Zuo Zhuan, which was compiled no later than 389 BC, and attributed to the blind 5th-century BC historian Zuo Qiuming.
	
	In ancient India, literature originated from stories that were originally orally transmitted. Early genres included drama, fables, sutras and epic poetry. Sanskrit literature begins with the Vedas, dating back to 1500–1000 BC, and continues with the Sanskrit Epics of Iron Age India.[47][48] The Vedas are among the oldest sacred texts. The Samhitas (vedic collections) date to roughly 1500–1000 BC, and the "circum-Vedic" texts, as well as the redaction of the Samhitas, date to c.1000‒500 BC, resulting in a Vedic period, spanning the mid-2nd to mid-1st millennium BC, or the Late Bronze Age and the Iron Age. The period between approximately the 6th to 1st centuries BC saw the composition and redaction of the two most influential Indian epics, the Mahabharata and the Ramayana, with subsequent redaction progressing down to the 4th century AD such as Ramcharitmanas.
	
	The earliest known Greek writings are Mycenaean (c.1600–1100 BC), written in the Linear B syllabary on clay tablets. These documents contain prosaic records largely concerned with trade (lists, inventories, receipts, etc.); no real literature has been discovered. Michael Ventris and John Chadwick, the original decipherers of Linear B, state that literature almost certainly existed in Mycenaean Greece, but it was either not written down or, if it was, it was on parchment or wooden tablets, which did not survive the destruction of the Mycenaean palaces in the twelfth century BC. Homer's epic poems, the Iliad and the Odyssey, are central works of ancient Greek literature. It is generally accepted that the poems were composed at some point around the late eighth or early seventh century BC. Modern scholars consider these accounts legendary. Most researchers believe that the poems were originally transmitted orally. From antiquity until the present day, the influence of Homeric epic on Western civilization has been significant, inspiring many of its most famous works of literature, music, art and film. The Homeric epics were the greatest influence on ancient Greek culture and education; to Plato, Homer was simply the one who "has taught Greece" – ten Hellada pepaideuken. Hesiod's Works and Days (c.700 BC) and Theogony are some of the earliest and most influential works of ancient Greek literature. Classical Greek genres included philosophy, poetry, historiography, comedies and dramas. Plato (428/427 or 424/423 – 348/347 BC) and Aristotle (384–322 BC) authored philosophical texts that are regarded as the foundation of Western philosophy, Sappho (c.630 – c.570 BC) and Pindar were influential lyric poets, and Herodotus (c.484 – c.425 BC) and Thucydides were early Greek historians. Although drama was popular in ancient Greece, of the hundreds of tragedies written and performed during the classical age, only a limited number of plays by three authors still exist: Aeschylus, Sophocles, and Euripides. The plays of Aristophanes (c.446 – c.386 BC) provide the only real examples of a genre of comic drama known as Old Comedy, the earliest form of Greek Comedy, and are in fact used to define the genre.
	
	The Hebrew religious text, the Torah, is widely seen as a product of the Persian period (539–333 BC, probably 450–350 BC). This consensus echoes a traditional Jewish view which gives Ezra, the leader of the Jewish community on its return from Babylon, a pivotal role in its promulgation. This represents a major source of Christianity's Bible, which has had a major influence on Western literature.
	
	The beginning of Roman literature dates to 240 BC, when a Roman audience saw a Latin version of a Greek play. Literature in Latin would flourish for the next six centuries, and includes essays, histories, poems, plays, and other writings.
	
	The Qur'an (610 AD to 632 AD),[69] the main holy book of Islam, had a significant influence on the Arab language, and marked the beginning of Islamic literature. Muslims believe it was transcribed in the Arabic dialect of the Quraysh, the tribe of Muhammad. As Islam spread, the Quran had the effect of unifying and standardizing Arabic.
	
	Theological works in Latin were the dominant form of literature in Europe typically found in libraries during the Middle Ages. Western Vernacular literature includes the Poetic Edda and the sagas, or heroic epics, of Iceland, the Anglo-Saxon Beowulf, and the German Song of Hildebrandt. A later form of medieval fiction was the romance, an adventurous and sometimes magical narrative with strong popular appeal.
	
	Controversial, religious, political and instructional literature proliferated during the European Renaissance as a result of the Johannes Gutenberg's invention of the printing press around 1440, while the Medieval romance developed into the novel.
	\subsection{Publishing}
	\paragraph{}
	Publishing became possible with the invention of writing but became more practical with the invention of printing. Prior to printing, distributed works were copied manually, by scribes.
	
	The Chinese inventor Bi Sheng made movable type of earthenware c.1045 and was spread to Korea later. Around 1230, Koreans invented a metal type movable printing. East metal movable type was spread to Europe between the late 14th century and early 15th century. In c.1450, Johannes Gutenberg invented movable type in Europe. This invention gradually made books less expensive to produce and more widely available.
	
	Early printed books, single sheets, and images created before 1501 in Europe are known as incunables or incunabula. "A man born in 1453, the year of the fall of Constantinople, could look back from his fiftieth year on a lifetime in which about eight million books had been printed, more perhaps than all the scribes of Europe had produced since Constantine founded his city in A.D. 330."
	
	\chapter{Study and criticism}
	\section{Literary theory}
	\paragraph{}
	A fundamental question of literary theory is "what is literature?" – although many contemporary theorists and literary scholars believe either that "literature" cannot be defined or that it can refer to any use of language.
	\section{Literary fiction}
	\paragraph{}
	Literary fiction is a term used to describe fiction that explores any facet of the human condition, and may involve social commentary. It is often regarded as having more artistic merit than genre fiction, especially the most commercially oriented types, but this has been contested in recent years, with the serious study of genre fiction within universities.
	
	The very best in literature is annually recognized by the Nobel Prize in Literature, which is awarded to an author from any country who has, in the words of the will of Swedish industrialist Alfred Nobel, produced "in the field of literature the most outstanding work in an ideal direction" (original Swedish: den som inom litteraturen har producerat det mest framstående verket i en idealisk riktning).
	
	\chapter{Types}
	\section{Poetry}
	\paragraph{}
	Poetry has traditionally been distinguished from prose by its greater use of the aesthetic qualities of language, including musical devices such as assonance, alliteration, rhyme, and rhythm, and by being set in lines and verses rather than paragraphs, and more recently its use of other typographical elements. This distinction is complicated by various hybrid forms such as digital poetry, sound poetry, concrete poetry and prose poem, and more generally by the fact that prose possesses rhythm. Abram Lipsky refers to it as an "open secret" that "prose is not distinguished from poetry by lack of rhythm".
	
	Prior to the 19th century, poetry was commonly understood to be something set in metrical lines: "any kind of subject consisting of Rhythm or Verses".Possibly as a result of Aristotle's influence (his Poetics), "poetry" before the 19th century was usually less a technical designation for verse than a normative category of fictive or rhetorical art. As a form it may pre-date literacy, with the earliest works being composed within and sustained by an oral tradition; hence it constitutes the earliest example of literature.
	
	\section{Prose}
	\paragraph{}
	As noted above, prose generally makes far less use of the aesthetic qualities of language than poetry. However, developments in modern literature, including free verse and prose poetry have tended to blur the differences, and poet T.S. Eliot suggested that while "the distinction between verse and prose is clear, the distinction between poetry and prose is obscure".[121] There are verse novels, a type of narrative poetry in which a novel-length narrative is told through the medium of poetry rather than prose. Eugene Onegin (1831) by Alexander Pushkin is the most famous example.
	
	On the historical development of prose, Richard Graff notes that, in the case of ancient Greece, "recent scholarship has emphasized the fact that formal prose was a comparatively late development, an 'invention' properly associated with the classical period".
	
	Latin was a major influence on the development of prose in many European countries. Especially important was the great Roman orator Cicero. It was the lingua franca among literate Europeans until quite recent times, and the great works of Descartes (1596 – 1650), Francis Bacon (1561 – 1626), and Baruch Spinoza (1632 – 1677) were published in Latin. Among the last important books written primarily in Latin prose were the works of Swedenborg (d. 1772), Linnaeus (d. 1778), Euler (d. 1783), Gauss (d. 1855), and Isaac Newton (d. 1727).
	\subsection{Novel}
	\paragraph{}
	A novel is a long fictional narrative, usually written in prose. In English, the term emerged from the Romance languages in the late 15th century, with the meaning of "news"; it came to indicate something new, without a distinction between fact or fiction. The romance is a closely related long prose narrative. Walter Scott defined it as "a fictitious narrative in prose or verse; the interest of which turns upon marvelous and uncommon incidents", whereas in the novel "the events are accommodated to the ordinary train of human events and the modern state of society". Other European languages do not distinguish between romance and novel: "a novel is le roman, der Roman, il romanzo", indicates the proximity of the forms.
	
	Although there are many historical prototypes, so-called "novels before the novel", the modern novel form emerges late in cultural history—roughly during the eighteenth century. Initially subject to much criticism, the novel has acquired a dominant position amongst literary forms, both popularly and critically.
	
	\subsection{Novella}
	\paragraph{}
	The publisher Melville House classifies the novella as "too short to be a novel, too long to be a short story". Publishers and literary award societies typically consider a novella to be between 17,000 and 40,000 words.
	
	\subsection{Short story}
	\paragraph{}
	A dilemma in defining the "short story" as a literary form is how to, or whether one should, distinguish it from any short narrative and its contested origin, that include the Bible, and Edgar Allan Poe.
	
	\subsection{Graphic novel}
	\paragraph{}
	Graphic novels and comic books present stories told in a combination of artwork, dialogue, and text.
	
	\subsection{Electronic literature}
	\paragraph{}
	Electronic literature is a literary genre consisting of works created exclusively on and for digital devices.
	
	\subsection{Non-fiction}
	\paragraph{}
	Common literary examples of non-fiction include, the essay; travel literature; biography, autobiography and memoir; journalism; letter; diary; history, philosophy, economics; scientific, nature, and technical writings.[8][137]
	
	Non-fiction can fall within the broad category of literature as "any collection of written work", but some works fall within the narrower definition "by virtue of the excellence of their writing, their originality and their general aesthetic and artistic merits".
	
	\section{Drama}
	\paragraph{}
	Drama is literature intended for performance.[139] The form is combined with music and dance in opera and musical theatre (see libretto). A play is a written dramatic work by a playwright that is intended for performance in a theatre; it comprises chiefly dialogue between characters. A closet drama, by contrast, is written to be read rather than to be performed; the meaning of which can be realized fully on the page.[140] Nearly all drama took verse form until comparatively recently.
	
	The earliest form of which there exists substantial knowledge is Greek drama. This developed as a performance associated with religious and civic festivals, typically enacting or developing upon well-known historical, or mythological themes,
	In the twentieth century, scripts written for non-stage media have been added to this form, including radio, television and film.
	
	\chapter{Bibliography}
	Bibliography
	Allan, Sarah (1991). The Shape of the Turtle: Myth, Art, and Cosmos in Early China. SUNY Press. ISBN 978-0-7914-0460-7.
	Baxter, William H. (1992). A Handbook of Old Chinese Phonology. Berlin: Mouton de Gruyter. ISBN 978-3-11-012324-1.
	Cuddon, J. A. (1998) [1977]. C.E. Preston (ed.). A Dictionary of Literary Terms and Literary Theory (4th rev. ed.). Oxford: Blackwell. ISBN 0-631-20271-4.
	Eagleton, Terry (2008). Literary Theory: An Introduction (Anniversary, 2nd ed.). Oxford: Blackwell Publishing. ISBN 978-1-4051-7921-8.
	Flood, Gavin (1996). An Introduction to Hinduism. Cambridge University Press. ISBN 978-0-521-43878-0.
	Flood, Gavin, ed. (2003). The Blackwell companion to Hinduism. Oxford: Blackwell Publ. ISBN 1-4051-3251-5.
	Hogan, P. Colm (2011). What Literature Teaches Us about Emotion. New York: Cambridge University Press.
	Foster, John Lawrence (2001). Ancient Egyptian Literature: An Anthology. Austin: University of Texas Press. p. xx. ISBN 978-0-292-72527-0.
	Frei, Peter (2001). "Persian Imperial Authorization: A Summary". In Watts, James (ed.). Persia and Torah: The Theory of Imperial Authorization of the Pentateuch. Atlanta, GA: SBL Press. p. 6. ISBN 9781589830158.
	Goody, Jack (1987). The Interface Between the Written and the Oral. Cambridge University Press. ISBN 978-0-521-33794-6.
	Goody, Jack (2006). "From Oral to Written: An Anthropological Breakthrough in Storytelling". In Franco Moretti (ed.). The Novel, Volume 1: History, Geography, and Culture. Princeton: Princeton UP. p. 18. ISBN 978-0-691-04947-2.
	Jacobs, Joseph (1888). The earliest English version of the Fables of Bidpai.
	MacDonell, Arthur Anthony (2004). A Practical Sanskrit Dictionary. Motilal Banarsidass. ISBN 978-81-208-2000-5.
	Paris, B.J. (1986). Third Force Psychology and the Study of Literature. Cranbury: Associated University Press.
	Preminger, Alex; et al. (1993). The New Princeton Encyclopedia of Poetry and Poetics. US: Princeton University Press. ISBN 978-0-691-02123-2.
	Radhakrishnan, S.; Moore, C. A. (1957). A Source Book in Indian Philosophy. Princeton, New Jersey: Princeton University Press. ISBN 978-0-691-01958-1.
	Romer, Thomas (2008). "Moses Outside the Torah and the Construction of a Diaspora Identity" (PDF). Journal of Hebrew Scriptures. 8, article 15: 2–12. Archived from the original (PDF) on 21 October 2020. Retrieved 27 September 2019.
	Ryder, Arthur W. (transl) (1925). The Panchatantra. University of Chicago Press. ISBN 81-7224-080-5.
	
	
\end{document}
